\chapter{Evaluation}
The main challenge of the project was understanding the data from a different domain to computer science, and how it needed to be interpreted, to be presented in an accessible form. This meant that the requirements only really became clear very late in the project, as the requirements were constantly adapting with my understanding of the data. 

Once the final requirements were decided upon, development went very smoothly. The end result is functional, and meets all of the achievable criteria, so I believe the design decisions were correct, and the implementation done in a clean and efficient manner. The codebase produced is only around 700 lines for the importation script and website combined, this isn't a terribly useful metric, but it does show that the code is at least concise.

The use of Node.js and MongoDB does set this project apart from older similar projects, that use Perl and PostgreSQL. One of the aims of this project was to demonstrate that the newer technologies can provide a tangible benefit over the older systems, and I think that the simplicity of the system, compared to it's levels of functionality do show that there is something to be gained from using a more modern stack. 

Once the website was deployed to the university servers, the researchers were able to use it for a more extended period, and appeared to be very happy with what was produced. They did make suggestions of additional features that could have been included, but these weren't discussed during the project so weren't really in scope. In the future though it would definitely be an interesting project to expand the system to include some of these features such as a comparison tool between genes, and a users system that would allow for administrators to upload new datasets.

The area that really could have been improved upon was the amount of time it took to get to understand the data. If the data had been better documented, less time would have had to be spent trying to understand it and there could have been more features developed that would have helped the researchers a lot more than what I was able to produce. Additionally there would have been more time to work on improving the test coverage of the application to improve it's reliability.

The application was built with very robust software engineering principles, and the code is a testament to how cleanly it was developed, there are very few smells\cite{smells} throughout the code. This is mostly thanks to the discipline enforced by having linters and automated testing. The only area that falls down in this regard is the amount of unit tests. As the project stands there are only a handful of unit tests to check the finding of coding sequences. Ideally it would be good to expand the testing to cover the Pug templates, the controller logic, and the search functionality. Unfortunately by the time the data was understood enough to develop the application fully, there was not enough time to test these features and write up this report. 

Future work on the project would focus on improving the test coverage of the unit tests, and then working on implementing a proper users system that would allow the website to be password protected on a user basis. This would fix the issue with the basic HTTP auth, enabling the site to work with the Varnish cache as well as be made accessible to the public with ease in the future. It is somewhat outside of the scope for this project, however it would be interesting to develop a tool that processes data from other analysis tools into the format ready for ingestion to the database. If such a tool were to be created this would allow this web service to become an open source option for making annotated genomes more accessible. 

I will continue to support this project after it's completion, and hopefully continue working with the researchers to develop new features and better the understanding of the data that has been produced, as well as potentially adding new datasets to the project for them. 



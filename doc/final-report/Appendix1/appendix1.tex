\chapter{Third-Party Code and Libraries}
\section{Nodestack}

\url{https://github.com/bag-man/nodestack}

This project uses boilerplate code developed by myself for previous projects. The code sets up a basic Node JS project, with access to to third party services. During the development of this project some features were merged back into the boilerplate, such as the mongoose integration.  This boilerplate makes use of several third party libraries, which are all listed in the package.json. Some extra modules have also been added specific to this project such as fasta2json and the clipboard module. 

\section{Externally hosted libraries}
\subsubsection*{JQuery} 

\url{https://github.com/jquery/jquery} - MIT

JQuery is a client side javascript library that makes navigating and interacting with the HTML DOM far easier. 

\subsubsection*{Bootstrap} 

\url{https://github.com/twbs/bootstrap} - MIT

Bootstrap is a library for styling webpages and making them easily responsive to different screen sizes.

\section{package.json}
Below is the modules listed in the package.json for the project, which can be found with the source code, all third party software is listed here along with it's version. This is how node modules are packaged, this automates the management of all modules, and locks the versions. For this project I was using yarn\cite{yarn} to manage the packages rather than the default node package manager (npm)\cite{npm}. Yarn has a couple of advantages, mainly that it is a lot faster than npm3 and that it automatically creates a lock file for modules and their dependencies. 

\subsection{Development Dependencies}
  This section of third party modules are not installed on the production builds of the application, they are only for use when developing the application, they don't change the functionality of it at all, just make it easier to work on. 

\subsubsection*{codecov} 

\url{https://www.npmjs.com/package/codecov} @ 1.0.1 - MIT

This provides integration with codecov's services, to provide interactive test coverage information.

\subsubsection*{basic-auth-connect} 

\url{https://www.npmjs.com/package/basic-auth-connect} @ 1.0.0 - MIT

This provides a basic HTTP auth middleware for express.

\subsubsection*{eslint}  

\url{https://www.npmjs.com/package/eslint} @ 2.3.0 - MIT

This is the linting engine that is used to check the source code for mistakes.

\subsubsection*{eslint-config-clock} 

\url{https://www.npmjs.com/package/eslint-config-clock} @ 1.2.0 - ISC

This is a set of configurations for eslint that describe the code formatting that is preferred for this project.

\subsubsection*{eslint-config-standard} 

\url{https://www.npmjs.com/package/eslint-config-standard} @ 5.1.0 - MIT

The clock config is built on top of this standard configuration

\subsubsection*{eslint-plugin-promise} 

\url{https://www.npmjs.com/package/eslint-plugin-promise} @ 3.3.0 - ISC

Eslint didn't support promises natively at the time of writing, so this was used to detect promises in the code.

\subsubsection*{eslint-plugin-standard} 

\url{https://www.npmjs.com/package/eslint-plugin-standard} @ 1.3.1 - MIT

Add's a few extra rules to eslints configuration options.

\subsubsection*{husky} 

\url{https://www.npmjs.com/package/husky} @ 0.13.2 - MIT

Simply runs the test suite before code is pushed to remote repositories.

\subsubsection*{istanbul} 

\url{https://www.npmjs.com/package/istanbul} @ 1.0.0-alpha.2 - BSD-3-Clause

Provides coverage information at the end of the test suite, and for codecov's usage.

\subsubsection*{mocha} 

\url{https://www.npmjs.com/package/mocha} @ 3.1.2 - MIT

Testing framework for javascript.

\subsubsection*{nodemon} 

\url{https://www.npmjs.com/package/nodemon} @ 1.11.0 - MIT

Monitors source files for changes, and relaunches the application when files are changed.

\subsection{Build Dependencies}
These third party modules are the frameworks libraries and modules that are actually used to by the application to function. 

\subsubsection*{babel}

\url{https://www.npmjs.com/package/babel} @ 7.5.2 - MIT

Transpiler for javascript. Used for converting ES6 code into ES5 for browser compatibility.

\subsubsection*{babel-core}

\url{https://www.npmjs.com/package/babel-core} @ 6.18.0 - MIT

Core configurations for babel.


\subsubsection*{babel-loader}

\url{https://www.npmjs.com/package/babel-loader} @ 6.2.5 - MIT

Allows for transpiling to be done from webpack build manager.

\subsubsection*{babel-preset-es2015}

\url{https://www.npmjs.com/package/babel-preset-es2015} @ 6.18.0 - MIT

The standard ES5 configuration.

\subsubsection*{clipboard}

\url{https://www.npmjs.com/package/clipboard} @ 1.6.1 - MIT

Clipboard module used for copying text to a users system clipboard from the browser.


\subsubsection*{express}

\url{https://www.npmjs.com/package/express} @ 4.14.0 - MIT

The web framework that powers the node application.

\subsubsection*{fasta2json}

\url{https://www.npmjs.com/package/fasta2json} @ 0.1.1 - MIT

Module that reads fasta files into JSON objects.

\subsubsection*{mongoose}

\url{https://www.npmjs.com/package/mongoose} @ 4.8.6 - MIT

MongoDB object modelling framework.

\subsubsection*{morgan}

\url{https://www.npmjs.com/package/morgan} @ 1.7.0 - MIT

Cleaner and clearer logging output.

\subsubsection*{pug}

\url{https://www.npmjs.com/package/pug} @ 2.0.0-beta11 - MIT

HTML templating language.

\subsubsection*{stylus}

\url{https://www.npmjs.com/package/stylus} @ 0.54.5 - MIT

CSS templating language.

\subsubsection*{webpack}

\url{https://www.npmjs.com/package/webpack} @ 2.2.1 - MIT

Javascript build manager.


% If you have made use of any third party code or software libraries, i.e. any code that you have not designed and written yourself, then you must include this appendix. 

% As has been said in lectures, it is acceptable and likely that you will make use of third-party code and software libraries. If third party code or libraries are used, your work will build on that to produce notable new work. The key requirement is that we understand what is your original work and what work is based on that of other people. 

% Therefore, you need to clearly state what you have used and where the original material can be found. Also, if you have made any changes to the original versions, you must explain what you have changed. 

% As an example, you might include a definition such as: 

% Apache POI library - The project has been used to read and write Microsoft Excel files (XLS) as part of the interaction with the client's existing system for processing data. Version 3.10-FINAL was used. The library is open source and it is available from the Apache Software Foundation 
% \cite{apache_poi}. The library is released using the Apache License 
% \cite{apache_license}. This library was used without modification. 

%\addcontentsline{toc}{chapter}{Development Process}
\chapter{Experiment Methods}

** Not sure what we put for this section **

This section should discuss the overall hypothesis being tested and justify the approach selected in the context of the research area.  Describe the experiment design that has been selected and how measurements and comparisons of results are to be made. 

You should concentrate on the more important aspects of the method. Present an overview before going into detail. As well as describing the methods adopted, discuss other approaches that were considered. You might also discuss areas that you had to revise after some investigation. 

You should also identify any support tools that you used. You should discuss your choice of implementation tools or simulation tools. For any code that you have written, you can talk about languages and related tools. For any simulation and analysis tools, identify the tools and how they are used on the project. 

For the parts of your project that need some engineering (hardware, software, firmware, or a mixture) to support the experiments, include details in your report about your design and implementation. You should discuss with your supervisor whether it is better to include a different top-level section to describe any engineering work.  In this template, Chapter 3 is suggested as a place for that discussion.

\section{Provided data}
The data for this project was produced by a contracted researcher who had moved onto another project, because of this there was limited opportunity to talk with them about the data, how it was produced and what needed doing to it. Unfortunately, the data did not include any form of documentation, or annotation as to how it was produced, the only metadata available was the filenames. 

This lead to a lot of confusion as it wasn't immediately obvious what each bit of data was and what it meant. Examining the data and learning about the biology behind it took several weeks for me to ascertain what bits of data were actually relevant to the project, and what other data needed producing. For each species there were three files that I would need to use. What was provided:

\begin{itemize}
  \item Raw assembled contigs of DNA, these would be used to get a genes position in the genome, and it's surrounding bases. 
  \item Protein sequences annotated with Gene Ontology ID's, these had been extracted from an NCBI nr blast
  \item Coding sequences, these are the nucleotide sequences that coded for the proteins. Uncertain as to how they were produced.
\end{itemize}

Linking this data to the Candida Genome Database was a key bit of information that was missing. To do this the annotated proteins for \textit{C. albicans}\cite{http://www.candidagenome.org/download/sequence/C_albicans_SC5314/Assembly22/current/C_albicans_SC5314_A22_current_default_protein.fasta.gz} were downloaded, then turned into a reference database with Diamond. With this the three species could then have their coding sequences aligned against \textit{C. albicans} to produce accession ID's from the Candida Genome Database.

There was then several sets of data required to map these proteins to the Candida Genome Database, that would allow the researchers to easily compare the genes to well annotated species. First was a file of GO annotations for all the proteins stored in the Candida Genome Database. \cite{http://www.candidagenome.org/download/go/gene_association.cgd.gz} This file conveniently also stores all the accession ID's for those proteins. This is what was used to map the alignment results to the Candida Genome Database.

In addition to this was a mapping file that mapped Candida Genome Database ID's to Uniprot ID's. With a small JavaScript script and some manual cleaning with a text editor, I was able to produce a complete mapping file in JSON, ready to be read in by the importation script. 

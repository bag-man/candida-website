\chapter{Background \& Objectives}

This section should discuss your preparation for the project, including background reading, your analysis of the problem and the process or method you have followed to help structure your work.  It is likely that you will reuse part of your outline project specification, but at this point in the project you should have more to talk about. 

\textbf{Note}: 

\begin{itemize}
   \item All of the sections and text in this example are for illustration purposes. The main Chapters are a good starting point, but the content and actual sections that you include are likely to be different.
   
   \item Look at the document on the Structure of the Final Report for additional guidance. 
   
\end{itemize}

\section{Background}
Researchers at Aberystwyth University have been studying three species of yeast, \textit{Candida tropicalis}, \textit{Candida boidinii}, and \textit{Candida shehatae}. To study the genetics of these species, they have each had their genomes sequenced by Illumina sequencing machines. From that point the sequences of DNA that have been read are then assembled into contigs (contiguous DNA fragments). This process produces an accurate representation of the species DNA, which then can then be aligned to other better understood DNA and annotated as such.

Arabinose and Xylose are five carbon sugars ubiquitously found in plants such as grass, which can be used as a feedstock for industrial biotechnology. \textit{Candida tropicalis} is able to convert arabinose \& xylose into arabitol and xylitol respectively. Xylitol is a commercially valuable anti-bacterial foodgrade sugar use in the manufacture of chewing gum. However arabitol and xylitol are stereoisomers and cannot be easily separated. \textit{Candida boidinii} cannot metabolise arabinose, for an unknown reason, but does metabolise Xylose but at a much slower rate which is commercially in-viable.

Understanding at a genomic level why \textit{Candida boidinii} is unable to utilise arabinose and genetically modifying \textit{Candida tropicalis} to have the same phenotype, may enable \textit{Candida topicalis} to exclusively produce xylitol and not arabitol. One other exciting possibility is using it as a low glycaemic index table sugar that can be used by diabetics.

What was your background preparation for the project? What similar systems or research techniques did you assess? What was your motivation and interest in this project? 

\section{Analysis}
Taking into account the problem and what you learned from the background work, what was your analysis of the problem? How did your analysis help to decompose the problem into the main tasks that you would undertake? Were there alternative approaches? Why did you choose one approach compared to the alternatives? 

There should be a clear statement of the research questions, which you will evaluate at the end of the work. 

In most cases, the agreed objectives or requirements will be the result of a compromise between what would ideally have been produced and what was felt to be possible in the time available. A discussion of the process of arriving at the final list is usually appropriate.

\section{Research Method and Software Process}
You need to describe briefly the life cycle model or research method that you used. You do not need to write about all of the different process models that you are aware of. Focus on the process model or research method that you have used. It is possible that you needed to adapt an existing method to suit your project; clearly identify what you used and how you adapted it for your needs.

For the research-oriented projects, there needs to be a suitable process for the construction of the software elements that support your work. 

\chapter{Background \& Objectives}

This section should discuss your preparation for the project, including background reading, your analysis of the problem and the process or method you have followed to help structure your work.  It is likely that you will reuse part of your outline project specification, but at this point in the project you should have more to talk about. 

\textbf{Note}: 

\begin{itemize}
   \item All of the sections and text in this example are for illustration purposes. The main Chapters are a good starting point, but the content and actual sections that you include are likely to be different.
   
   \item Look at the document on the Structure of the Final Report for additional guidance. 
   
\end{itemize}

\section{Background}
Researchers at Aberystwyth University have been studying three species of yeast, \textit{Candida tropicalis}, \textit{Candida boidinii}, and \textit{Candida shehatae}. To study the genetics of these species, they have each had their genomes sequenced by Illumina sequencing machines. From that point the sequences of DNA that have been read are then assembled into contigs (contiguous DNA fragments). This process produces an accurate representation of the species DNA, which then can then be aligned to other better understood DNA and annotated as such.

Arabinose and Xylose are five carbon sugars ubiquitously found in plants such as grass, which can be used as a feedstock for industrial biotechnology. \textit{Candida tropicalis} is able to convert arabinose \& xylose into arabitol and xylitol respectively. Xylitol is a commercially valuable anti-bacterial foodgrade sugar use in the manufacture of chewing gum. However arabitol and xylitol are stereoisomers and cannot be easily separated. \textit{Candida boidinii} cannot metabolise arabinose, for an unknown reason, but does metabolise Xylose but at a much slower rate which is commercially in-viable.

Understanding at a genomic level why \textit{Candida boidinii} is unable to utilise arabinose and genetically modifying \textit{Candida tropicalis} to have the same phenotype, may enable \textit{Candida topicalis} to exclusively produce xylitol and not arabitol. One other exciting possibility is using it as a low glycaemic index table sugar that can be used by diabetics.

What was your background preparation for the project? 

* Looked at databases such as CHADO

Before undertaking this project -->I<-- had some experience working with genomic data, and had been sitting in on lectures for a functional genomics module. This helped me get a good understanding of the basics of genetics and get a foothold on the terminology that is used in the world of bioinformatics.  As the core of the project focuses on having the data ingested into a database, my first action was to investigate the current database solutions that had been developed for genomic information. 

The most well established was CHADO\ref{chado} which was initially developed back in 2003 for the xyz project. It has since then grown to accomdate genomic information for eukaryotes, plants, and other complex multicell animals. It uses PostgreSQL to store it's data and Perl to setup and maintain the database. Due to it's monolithic approach of being a one size fits all solution, it has over 200 tables in a very complex schema. This makes working on it rather difficult, as you have to consider a huge amount of relations between your tables. 

When installing and populating CHADO, it was clear that the project wasn't very well maintained as during the installation several of the Perl modules that it depends on where failing their tests and not installing. This meant I had to manually download and fix several Perl modules just to get the application to install correctly. If it takes an experienced programmer and linux admin several hours to follow the default installation instructions, debugging issues at several steps, it isn't going to be appropriate to impose such a system on potentially less technically minded Biologists. This combined with the fact that Perl has now been superseded by more modern scripting languages such as Python and JavaScript, thanks to their superior package management and syntax, it was clear that a more modern approach would be more sustainable and maintainable for this project and for the future. 

* Looked into blast / diamond / GPU / blas2go

As the source of the data was uncertain, it was decided to investigate performing alignments in the event that more data, or different data was needed for the project than what was provided. The original alignment tool BLAST\ref{blast} was developed in the 90's and is very widely accepted as the de facto standard for performing alignments. I should talk about what an alignment is. As alignment is a computationally intensive task there have been many efforts to improve the efficiency of the algorithm and make the most of developments made in computing since the 90's. One area that has been pursued is the use of GPU's to perform alignments in parallel utilising hundreds or thousands of cores of the GPU, compared to the tens of cores on a modern CPU. 

Make my blog post into something proper http://blog.owen.cymru/project-log-today-i-found-out-rm-can-only-take-100-000-arguments-at-a-time/


What similar systems or research techniques did you assess? 

From studying bioinformatics and researching the most common methods of manipulating and analysing the data gathered, a trend was found that indicated that unlike computer science, where tools are highly developed and have evolved to a very stable state where the best tools are known and highly specialised to suit their purpose, in bioinformatics a combination of factors including the relative youth of the field and the large number of competing projects that don't collaborate; there have been a wide array of tools created, often by biologists with no software engineering experience, none of which have been adopted and honed as the de facto standard. This means that for every task in the bioinformatics space there is often many different solutions offered, to what can be a very simple problem. 

What was your motivation and interest in this project? 

My interest in this project stems from my interest in genetics, something that has always fascinated me since I learned about how we evolved into being. As a computer scientist I relish the chance to apply the knowledge of my domain, to a real life application that can have a measurable positive impact on the world. Hopefully I will be able to use this experience to further my career into bioinformatics as well as helping contribute to the research being done. 

\section{Analysis}
Taking into account the problem and what you learned from the background work, what was your analysis of the problem? 

From researching the current solutions to the problem of annotating, storing and presenting genetic information, it was clear that a modern solution (describe that) wasn't currently in the public open source domain. Of the open source projects that were currently in use the majority appeared to be very old and monolithic, commonly written in Perl. Expand on all this etc.

How did your analysis help to decompose the problem into the main tasks that you would undertake?

Looking at the data that I was provided with there was three clear stages to the development of my solution. The first would be to ensure that the data I had was valid and in the same format for each of the species that were being analysed. The next was to import that data into a database, and then from there develop a web front end to interact with the data in the database. 

Were there alternative approaches? 

Use one of the existing databases and web front ends, but eww.

Why did you choose one approach compared to the alternatives? 

Diamond or blast2go?
SQL vs MongoDB Discuss in many details

There should be a clear statement of the research questions, which you will evaluate at the end of the work. 

In most cases, the agreed objectives or requirements will be the result of a compromise between what would ideally have been produced and what was felt to be possible in the time available. A discussion of the process of arriving at the final list is usually appropriate.

\section{Research Method and Software Process}
You need to describe briefly the life cycle model or research method that you used. You do not need to write about all of the different process models that you are aware of. Focus on the process model or research method that you have used. It is possible that you needed to adapt an existing method to suit your project; clearly identify what you used and how you adapted it for your needs.

For the research-oriented projects, there needs to be a suitable process for the construction of the software elements that support your work. 

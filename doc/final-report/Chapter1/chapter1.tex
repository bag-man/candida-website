\chapter{Background \& Objectives}

This section should discuss your preparation for the project, including background reading, your analysis of the problem and the process or method you have followed to help structure your work.  It is likely that you will reuse part of your outline project specification, but at this point in the project you should have more to talk about. 

\textbf{Note}: 

\begin{itemize}
   \item All of the sections and text in this example are for illustration purposes. The main Chapters are a good starting point, but the content and actual sections that you include are likely to be different.
   
   \item Look at the document on the Structure of the Final Report for additional guidance. 
   
\end{itemize}

\section{Background}
Researchers at Aberystwyth University have been studying three species of yeast, \textit{Candida tropicalis}, \textit{Candida boidinii}, and \textit{Candida shehatae}. To study the genetics of these species, they have each had their genomes sequenced by Illumina sequencing machines. From that point the sequences of DNA that have been read are then assembled into contigs (contiguous DNA fragments). This process produces an accurate representation of the species DNA, which then can then be aligned to other better understood DNA and annotated as such.

Arabinose and Xylose are five carbon sugars ubiquitously found in plants such as grass, which can be used as a feedstock for industrial biotechnology. \textit{Candida tropicalis} is able to convert arabinose \& xylose into arabitol and xylitol respectively. Xylitol is a commercially valuable anti-bacterial foodgrade sugar use in the manufacture of chewing gum. However arabitol and xylitol are stereoisomers and cannot be easily separated. \textit{Candida boidinii} cannot metabolise arabinose, for an unknown reason, but does metabolise Xylose but at a much slower rate which is commercially in-viable.

Understanding at a genomic level why \textit{Candida boidinii} is unable to utilise arabinose and genetically modifying \textit{Candida tropicalis} to have the same phenotype, may enable \textit{Candida topicalis} to exclusively produce xylitol and not arabitol. This would mean a cheap and easy to produce source of Xylitol, which can be used for many applications. One exciting possibility is using it as a low glycaemic index table sugar that can be used by diabetics.

\section{Existing Databases}

Before undertaking this project -->I<-- had some experience working with genomic data, and had been sitting in on lectures for a functional genomics module. This helped me get a good understanding of the basics of genetics and get a foothold on the terminology that is used in the world of bioinformatics.  As the core of the project focuses on having the data ingested into a database, the first action was to investigate the current database solutions that had been developed for genomic information. 

The most well established was CHADO\ref{chado} which was initially developed back in 2005\cite{http://flybaselb1.uits.indiana.edu/reports/FBrf0206380.html} for the FlyBase\cite{flybase} project. It has since then grown to accommodate genomic information for eukaryotes, plants, and other complex multicell animals. It uses PostgreSQL\cite{postgres} to store it's data and Perl\cite{perl} to setup and maintain the database. Due to it's monolithic approach of being a one size fits all solution, it has over 200 tables in a very complex schema. This makes working on it rather difficult, as you have to consider a huge amount of relations between your tables. 

When installing and populating CHADO, it was clear that the project wasn't very well maintained as during the installation several of the Perl modules that it depends on where failing their tests and not installing. This meant several modules had to be manually fixed just to get the application to install correctly. If it takes an experienced programmer and linux admin several hours to follow the default installation instructions, debugging issues at several steps, it isn't going to be appropriate to impose such a system on potentially less technically minded Biologists. This combined with the fact that Perl has now been superseded by more modern scripting languages such as Python\cite{python} and JavaScript\cite{javascript}, thanks to their superior package management, eco-systems and syntax, it was clear that a more modern approach would be more sustainable and maintainable for this project and for the future. 

\section{Alignment Tools}

As the source of the data was uncertain, it was decided to investigate performing alignments in the event that more data, or different data was needed for the project than what was provided. The original alignment tool BLAST\ref{blast} was developed in the 90's and is very widely accepted as the de facto standard for performing alignments. Alignment is when a DNA, or protein sequence is compared against a database of known protein sequences that have already been determined via previous research. This allows a researcher to match their sequences against known proteins and annotate them accordingly. These annotations aren't proof that the gene codes for that exact protein, but it is a strong indicator that it does. To prove it the researchers will test it in a lab experiment, however as there are many thousands of potential genes to test it is a lot more efficient to test genes that are predicted via an alignment. An alignment is a computationally intensive task there have been many efforts to improve the efficiency of the algorithm and make the most of developments made in computing since the 90's. One area that has been pursued is the use of GPU's to perform alignments in parallel utilising hundreds or thousands of cores of the GPU, compared to the tens of cores on a modern CPU. 

Initially I tested Diamond\cite{diamond} and was able to blast the data sets against the NCBI\cite{ncbi} non-redundant protein database, in about an hour and a quarter for each species, this was only using the CPU and may have been bottlenecked by the mechanical drives being used as storage. The hardware being used for this was an Intel i7-6700k @ 4.6Ghz, 16GB of RAM, a 7200rpm Hard Disk Drive, and a Nvidia GTX 980. As there was a high performace graphics card available it would be advantageous to make use a GPU accelerated alignment tool to make the most of the hardware available, and reduce the time that any future alignments may take. 

The first attempt to use a GPU powered tool was BarraCUDA\cite{barracuda}, which installed successfully and appeared to run fine, however it would attempt to read in the entire reference database into memory. This was an issue as the NCBI non-redundant database is around 66GB's in size, and the system that was being used only had 16GB's of RAM available. 

Looking for other options I encountered CLAST\cite{clast}, which unfortunately had compilation errors on the system. Debugging this was out of scope for my project as in this initial stage it didn't seem necessary. Next to try was Cudasw++\cite{cudasw} this did install, but ran into the same issues as Barracuda, being limited by the system memory available. 

In an attempt to get GPU acceleration working, the NCBI nr database was split up into 10GB chunks for use with these tools, however even after being reduced in size, BarraCUDA and Cudasw++ both had errors and failed to complete an alignment. It was decided that although the speed gains of a GPU accelerated alignment tool would be significant, the Diamond alignment was quick enough to not be a major bottleneck for the project, when compared with the extra time and effort that would have to be put into investigating how to get the GPU alignments to work. 

\section{Similar systems}

From studying bioinformatics and researching the most common methods of manipulating and analysing the data gathered, a trend was found that indicated that unlike computer science, where tools are highly developed and have evolved to a very stable state where the best tools are known and highly specialised to suit their purpose, in bioinformatics a combination of factors including the relative youth of the field and the large number of competing projects that don't collaborate; there have been a wide array of tools created, often by biologists with no software engineering experience, none of which have been adopted and honed as the de facto standard. This means that for every task in the bioinformatics space there is often many different solutions offered, to what can be a very simple problem. 

Gbrowse?

What was your motivation and interest in this project? 

My interest in this project stems from my interest in genetics, something that has always fascinated me since I learned about how we evolved into being. As a computer scientist I relish the chance to apply the knowledge of my domain, to a real life application that can have a measurable positive impact on the world. Hopefully I will be able to use this experience to further my career into bioinformatics as well as helping contribute to the research being done. 

\section{Analysis}
Taking into account the problem and what you learned from the background work, what was your analysis of the problem? 

From researching the current solutions to the problem of annotating, storing and presenting genetic information, it was clear that a modern solution (describe that) wasn't currently in the public open source domain. Of the open source projects that were currently in use the majority appeared to be very old and monolithic, commonly written in Perl. Expand on all this etc.

How did your analysis help to decompose the problem into the main tasks that you would undertake?

Looking at the data that I was provided with there was three clear stages to the development of my solution. The first would be to ensure that the data I had was valid and in the same format for each of the species that were being analysed. The next was to import that data into a database, and then from there develop a web front end to interact with the data in the database. 

Were there alternative approaches? 

An alternative approach to this project would have been to use one of the existing databases such as Intermine or CHADO, and then use a web front end compatible with those databases such as GBrowse or Jbrowse. This would like require slight changes to how the data was processed before going into these established systems, however it would mean that the data would be in a format that is well understood by bioinformaticians. 

Why did you choose one approach compared to the alternatives? 

As a computer scientist looking at this problem, there doesn't seem to be much need for a very complex solution. Once the data is generated, the hard bit, it just needs to be made into a database friendly format and then ingested to the database system. From there building a simple web front end to make the database accessible is a very simple task. There isn't any need to manipulate or modify the data, simply create and read it. Because of this it seemed that the current systems were greatly over complicating matters. 

SQL vs MongoDB Discuss in many details
Diamond or blast2go?

There should be a clear statement of the research questions, which you will evaluate at the end of the work. 

In most cases, the agreed objectives or requirements will be the result of a compromise between what would ideally have been produced and what was felt to be possible in the time available. A discussion of the process of arriving at the final list is usually appropriate.

\section{Research Method and Software Process}
You need to describe briefly the life cycle model or research method that you used. You do not need to write about all of the different process models that you are aware of. Focus on the process model or research method that you have used. It is possible that you needed to adapt an existing method to suit your project; clearly identify what you used and how you adapted it for your needs.

For the research-oriented projects, there needs to be a suitable process for the construction of the software elements that support your work. 

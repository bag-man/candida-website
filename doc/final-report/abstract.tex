\thispagestyle{empty}

\begin{center}
    {\LARGE\bf Abstract}
\end{center}

Understanding data produced by next gen sequencing technologies is a difficult problem, not helped by a lack of well designed software and user experiences. 

Researchers at Aberystwyth University are looking into the possibly of modifying a species of yeast, \textit{Candida Tropicalis}, to convert xylose into xylitol. 

If xylitol can be produced by \textit{C. tropicalis} it would provide a cheaper supply of the sugar, which has antibacterial properties, as well as a very low glycemic index, making it suitable for diabetics.

To do this they have sequenced the DNA of three species of yeast and need an accessible and meaningful way to interact with the data produced. 

Most genome database software is already over a decade old, and using very out of date practices, and technologies. 

The aim for this project is to apply modern web development standards to this problem and produce a web site to display their data using Node JS, with the data stored in a NoSQL database, something that hasn't been done before. 

The implementation of this database and website has shown that NoSQL is a viable approach to storing and viewing genomic information, especially for smaller genomes. However it does also act as a proof of concept and provide a starting step for the production of a larger open source project that aims to accommodate larger genomes. 

Importantly it shows that modern technologies and approaches can have a meaningful benefit over the old monolithic solutions, especially in terms of producing high quality maintainable software.

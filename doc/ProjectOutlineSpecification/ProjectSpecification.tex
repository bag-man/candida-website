\documentclass[11pt,fleqn,twoside]{article}
\usepackage{makeidx}
\makeindex
\usepackage{palatino} %or {times} etc
\usepackage{plain} %bibliography style
\usepackage{amsmath} %math fonts - just in case
\usepackage{amsfonts} %math fonts
\usepackage{amssymb} %math fonts
\usepackage{lastpage} %for footer page numbers
\usepackage{fancyhdr} %header and footer package
\usepackage{mmpv2}
\usepackage{url}
\usepackage{hyperref}

% the following packages are used for citations - You only need to include one.
%
% Use the cite package if you are using the numeric style (e.g. IEEEannot).
% Use the natbib package if you are using the author-date style (e.g. authordate2annot).
% Only use one of these and comment out the other one.
\usepackage{cite}
%\usepackage{natbib}

\begin{document}

\name{Owen Garland}
\userid{owg1}
\projecttitle{Building an online resource for Candida Tropicalis}
\projecttitlememoir{Building an online resource for Candida Tropicalis} %same as the project title or abridged version for page header
\reporttitle{Outline Project Specification}
\version{0.1}
\docstatus{Draft}
\modulecode{CS39440}
\degreeschemecode{G600}
\degreeschemename{Software Engineering}
\supervisor{Wayne Aubrey} % e.g. Neil Taylor
\supervisorid{waa2}
\wordcount{}

%optional - comment out next line to use current date for the document
%\documentdate{10th February 2014}
\mmp

\setcounter{tocdepth}{3} %set required number of level in table of contents


%==============================================================================
\section{Project description}
%==============================================================================
\textit{Candida Tropicalis} is a yeast, that has had its genome sequenced by researchers in Aberystwyth. The researchers are hoping to edit some of the genes in the yeast to enable it to ferment sugars found in grass into xylan, that would then be able to create xylitol; which can be used in the manufacture of many products including chewing gum. 

To do this they plan to compare the genes found in Tropicalis with two other similar species of yeast, to see where they differ and to see if they can be engineered to produce these sugars. 

The data is currently in the form of the assembled contigs (strands of sequenced DNA). This is useful for the work they are doing, but it means that when they want to inspect a gene they need to analyse the data and annotate for each individual gene. The main aim for this project will be to produce an annotated dataset for each of the three species of yeast and then compare them them to reveal the differences. 

Once this is done I can then work on making this data more accessible to the researchers by building a web resource that will allow them to browse and compare the found genes. Eventually this resource would be made public for others to see. 


%==============================================================================
\section{Proposed tasks}
%==============================================================================
There are several steps involved in annotating the genome for \textit{Candida Tropicalis}, the first stage will be to find the open reading frames (ORF's), these are the beginning and end markers for a gene. The start of a gene is marked with an ATG sequence, and ended with a TAA or TGA sequence. There have been tools developed to perform this task, the most develop too appears to be OrfM\cite{OrfM}. This tool will produce data that can then be annotated using blast. 

Diamond will allow us to compare the genes found in \textit{Candida Tropicalis} with a reference database of an already annotated genome. From there it will produce a set of data that has the genomes labelled with the matching genes from the reference database.

Once the genome has been annotated I will be putting the data into a database, this might be Chado\cite{chado} which is an established schema for storing genetic data, it would be interesting to explore the possibilities of using a NoSQL database such as MongoDB for the project. 

Once the data is in a database, the next stage will be developing a web application that allows users to browse the genome. For this I will be using NodeJS, as I won't have much time left to build the website I need to use tools that I am familiar with and have a lot of boilerplate written for me already. 


%==============================================================================
\section{Project deliverables}
%==============================================================================
At the end of this project I will aim to provide the following deliverables:

\begin{itemize}
  \item Annotated genome of \textit{Candida Tropicalis}
  \item Database containing the annotated data
  \item Web accessible gene browser
\end{itemize}

%
% Start to comment out / remove the following lines. They are only provided for instruction for this example template.  You don't need the following section title, because it will be added as part of the bibliography section.
%
% End of comment out / remove the lines. They are only provided for instruction for this example template.
%


\nocite{*} % include everything from the bibliography, irrespective of whether it has been referenced.

% the following line is included so that the bibliography is also shown in the table of contents. There is the possibility that this is added to the previous page for the bibliography. To address this, a newline is added so that it appears on the first page for the bibliography.
\newpage
\addcontentsline{toc}{section}{Initial Annotated Bibliography}
%
% example of including an annotated bibliography. The current style is an author date one. If you want to change, comment out the line and uncomment the subsequent line. You should also modify the packages included at the top (see the notes earlier in the file) and then trash your aux files and re-run.
%\bibliographystyle{authordate2annot}
\bibliographystyle{IEEEannot}
\renewcommand{\refname}{Annotated Bibliography}  % if you put text into the final {} on this line, you will get an extra title, e.g. References. This isn't necessary for the outline project specification.
\bibliography{mmp} % References file

\end{document}
